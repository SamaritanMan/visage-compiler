\documentclass{article}
\usepackage{semantic}

\reservestyle{\operator}{\textbf}
\reservestyle{\keyword}{\texttt}
\keyword{Null,Void,Object,Integer,Number,Boolean,String,Duration}
\keyword{Int8,Int16,UInt16,Int32,Int64,Float32,Float64,IntXx,FloatXx}
\keyword{Byte,Short,Character,Integer,Long,Float,Double,int,float}
\keyword{null,true,false}
\keyword{class,extends,function,Required,Optional,Sequence,JavaArray,JavaXxx}
\operator{req,opt,seq,optof,isopt,arr,sizeof,default}
\mathlig{!=}{\mathrel{\texttt{!=}}}

\newcommand{\convertsto}{\mathrel{\tilde{\rightarrow}}}
\newcommand{\req}{\<req>\;}
\newcommand{\opt}{\<opt>\;}
\newcommand{\seq}{\<seq>\;}
\newcommand{\arr}{\<arr>\;}
\newcommand{\optof}{\<optof>\;}
\newcommand{\isopt}{\<isopt>\;}
\newcommand{\sizeof}{\<sizeof>\;}
\newcommand{\default}{\<default>\;}
\newcommand{\alt}{\;|\;}
\newcommand{\sqb}{\texttt{[]}}
\newcommand{\class}[3]{\<class> \; #1 \; \<extends> \; \bar{#2} \; \{ #3 \} }

\begin{document}

\section{The abstract type system}

For sake of clarity, we first develop an an \emph{abstract type
system} for JavaFX, whose syntax is divorced from the surface syntax
of the JavaFX Script language.  After defining the typing rules in the
abstract type system, we then map the JavaFX language syntax to the
abstract type system to form the \emph{user type system}.  Not all
types in the abstract type system are necessarily denotable or
expressible in the user type system.

\subsection{Ground types and cardinalities}

Most JavaFX types are built from a combination of an element type and
a ``cardinality''; the useful cardinalities are ``exactly one'' (a
\emph{required} type), ``zero or one'' (an \emph{optional} type), and
``zero or more'' (a \emph{sequence} type).  The element type defines
the type of the elements that a variable of that type may take on.  We
use the term \emph{ground type} to define a type that can be combined
with a cardinality to yield a JavaFX type.  All Java classes are valid
as \emph{ground types}, as are all JavaFX classes, as well as the
predefined classes \<Int32>, \<Float64>, and \<Boolean>.  The unary
cardinality operators (or type constructors) \<req>, \<opt>, and
\<seq>, define required, optional, and sequence type respectively.

Required types act much like the primitive types in Java; they cannot
be null.  Optional types act much like reference types in Java; they
either hold a reference to an instance of the element type, or they
hold null.  Sequence types hold an ordered set of zero or more non-null values.  
Sequence elements can be extracted from a sequence value using the familiar
$x[e]$ notation.  The empty sequence is indicated by a null value, $\sqb$ (see section \ref{sec:null}.)  

Function types are identified with the usual arrow notation ($T ->
S$); for simplicity we consider only unary functions.  Higher-arity
functions are handled by currying; nullary functions will be handled
by introducing a \<Void> type.

The metavariables $A$, $B$, and $C$ (and their derivatives) range over
class names (whether defined in Java or JavaFX, including interfaces).  Where
they appear with a $J$ subscript ($A_J$), they represent Java classes; where they
appear with an $F$ subscript, they represent JavaFX classes.  
The metavariables $E$, $F$, and $G$ range over the ground types.  The
metavariables $S$, $T$, $U$, and $V$ range over all types.

\begin{figure}[htpb]
\begin{eqnarray*}
     P &::=& \<Int32> \alt \<Float64> \alt \<Boolean> \\
     G &::=& P \alt C \alt T -> S \\
     T &::=& \req G
           \alt \opt G
           \alt \seq G 
           \alt \optof T \\
     L &::=& \class{A_{F}}{B}{ \bar{T}\; \bar{f} } \\
     e &::=& x \alt f(e) \alt x[e] \alt x.y \alt \sizeof x \alt \<true> \alt \<false> 
\end{eqnarray*}
\caption{Basic type syntax}
\label{basic-type-syntax}
\end{figure}

The notation $\bar{A}$ in the syntax rules and typing judgements
indicates a list of zero or more elements that would match the
metavariable $A$; where $\bar{A}$ appears, the elements of $\bar{A}$
are identified as $A_i$.  It is used, for example, in the syntax for
defining a class to indicate that a JavaFX class may have multiple
superclasses.  When it appears in the conclusion of a typing
judgement, it means that the conclusion is true for all the elements
that match the metavariable $A$.

We do not yet treat Java primitives in this section; we assume only
the existence of the ground types \<Int32>, \<Float64>, and
\<Boolean>.  The basic syntax for defining types in the abstract type
system is shown in figure \ref{basic-type-syntax}, and the basic
typing rules are shown in figure \ref{basic-type-rules}.  The
metafunction $fields$ is defined as in Featherweight Java; it is a set
of $(name, type)$ pairs describing the instance variables of a class.
Similarly, the metafunction $methods$ describes the methods of a
class.  (TODO: Provide formal definitions.)

It is also desirable to be able to take a type and convert it to
a type of another cardinality.  The type operator \<optof> yields a type that is
of the same ground type but with an optional cardinality, as follows:

\begin{eqnarray*}
\optof \req G &=& \opt G \\
\optof \opt G &=& \opt G \\
\optof \seq G &=& \seq G \\
\end{eqnarray*}

\begin{figure}[htpb]

\[
\inference[\textsc{T-Idx}]{
  \Gamma |- x : \seq G & \Gamma |- t : \opt \<Int32>
}{
  \Gamma |- x[t] : \opt G
}
\]

\[
\inference[\textsc{T-App-Opt}]{
  \Gamma |- f : \opt (T -> S) & \Gamma |- x : T
}{
  \Gamma |- f(x) : \optof S
}
\]
\[
\inference[\textsc{T-App}]{
  \Gamma |- f : \req (T -> S) & \Gamma |- x : T
}{
  \Gamma |- f(x) : S
}
\]

\[
  \inference[\textsc{T-Field}]{
    \Gamma |- e_0 : \req C_0 &
    \mathit{fields}(C_0) = \bar{C} \; \bar{f}
  }{
    \Gamma |- e_0.f_i : C_i
  }
\]
\[
  \inference[\textsc{T-Field-Opt}]{
    \Gamma |- e_0 : \opt C_0 &
    \mathit{fields}(C_0) = \bar{C} \; \bar{f}
  }{
    \Gamma |- e_0.f_i : \optof C_i
  }
\]

\begin{eqnarray*}
\sizeof x &:& \<Int32> \\
\<true> &:& \<Boolean> \\
\<false> &:& \<Boolean> \\
\end{eqnarray*}

\caption{Basic typing rules}
\label{basic-type-rules}
\end{figure}

\subsection{Null}
\label{sec:null}

The special type \<Null> has only the single distinguished value
\<null>.  In JavaFX, as in SQL, \<null> represents a nonexistent
value.  The special type $\<Null>\sqb$ has only the single distinguished
value $\sqb$, which is equal to \<null> but has a different static type.  
Figure \ref{null-typing} shows the syntax and typing
assertions associated with adding \<Null> to the type system.

\begin{figure}[htpb]
\begin{eqnarray*}
     T &::=& \dots 
             \alt \<Null> \alt \<Null>\sqb \\
     e &::=& \dots 
             \alt \<null> \alt \sqb \\
\end{eqnarray*}
\begin{eqnarray*}
    \<Null> &<:& \<Null>\sqb \\
    \<Null> &<:& \opt G \\
    \<Null>\sqb &<:& \seq G \\
    \optof \<Null> &=& \<Null> \\
    \optof \<Null>\sqb &=& \<Null>\sqb \\
    \<null> &:& \<Null> \\
    \sqb &:& \<Null>\sqb \\
\end{eqnarray*}

\caption{Null}
\label{null-typing}
\end{figure}


\subsubsection{Null conversions}
\label{sec:null-conversions}

Optional types can be converted to their corresponding required type
by coercing null values to a \emph{default value} for the
corresponding ground type.  The default values are defined by:

\begin{eqnarray*}
  \default \<IntXx> &=& 0 \\
  \default \<FloatXx> &=& 0.0 \\
  \default \<Boolean> &=& \<false> \\
  \default \<String>  &=& \textrm{``''} \\
  \default \<Duration>  &=& 0ms \\
  \default G  &=& \<null> \textrm{ (otherwise)}  \\
\end {eqnarray*}

We use the operator $\convertsto$ to indicate that one type is
convertible to another; type conversions come into play when subtyping
fails.  We can statically convert $\opt G$ to $\req
G$ as needed so long as $G$ has a non-null default value:

\[
\inference{\default G != \<null>}{\opt G \convertsto \req G}
\]

The conversion algorithm is straightforward; if the value is \<null>,
it is replaced with the default value for that type, otherwise the
value is simply interpreted as a value of the required type.  This
conversion is needed because many expressions that ``usually'' yield a
non-null value but are of optional type (such as sequence indexing,
member selection, and function invocation) may sometimes yield \<null>.
In the following program:

\begin{verbatim}
var x : Integer[] = [1, 2, 3];
var y : Integer = x[23];
\end{verbatim}

The type of $y$ is $\req \<Int32>$, but the type of $x[23]$ is $\opt
\<Int32>$ and it evaluates to \<null>.  The right-hand side is evaluated,
yielding \<null>; when assigned to $y$, it is converted to the default
type for $\<Int32>$, which is zero.

We can also dynamically convert $\seq G$ to $\opt G$ when the size of
the sequence is zero or one, throwing a runtime exception if the size
of the sequence is greater than one.


\subsection{Object}

The Java type \texttt{java.lang.Object} (referred to simply as
\<Object>) is significant, though does not form a ``top'' type as it
does in Java; instead each cardinality of \<Object> forms a ``top''
type for the class hierarchy for that cardinality:

\begin{eqnarray*}
    \req G &<:& \req \<Object> \\
    \opt G &<:& \opt \<Object> \\
    \seq G &<:& \seq \<Object> \\
\end{eqnarray*}


\subsection{Subtyping}

Figure \ref{basic-subtype} shows the subtyping rules for the types
defined in Figure \ref{basic-type-syntax}.  Rules \textsc{S-Refl},
\textsc{S-Trans}, \textsc{S-Cov}, and \textsc{T-Sub} are standard, and
borrowed from $\lambda_{<:}$ (the simply typed lambda calculus with
subtyping).  \textsc{S-Java} shows how Java subtyping maps to JavaFX
subtyping; \textsc{S-Card} and \textsc{S-Extends} show the subtyping
relationships for cardinality types and class types in JavaFX.  

\begin{figure}[htpb]
\[ \inference[\textsc{S-Refl}]{}{T <: T}
\]

\[ 
\inference[\textsc{S-Trans}]{
T <: S & S <: U
}{
T <: U
}
\]

\[ 
\inference[\textsc{S-Arrow}]{
T <: S & U <: V
}{
\req (V -> T) <: \req (U -> S) 
}
\]
\[ 
\inference[\textsc{S-Arrow-Opt}]{
T <: S & U <: V
}{
\opt (V -> T) <: \opt (U -> S) 
}
\]
\[ 
\inference[\textsc{S-Arrow-Seq}]{
T <: S & U <: V
}{
\seq (V -> T) <: \seq (U -> S) 
}
\]

\[ 
\inference[\textsc{T-Sub}]{
\Gamma |- t : S & S <: T
}{
\Gamma |- t : T
}
\]

\[ \inference[\textsc{S-Card}]{}{\req G <: \opt G <: \seq G}
\]

\[
  \inference[\textsc{S-Java}]{
    A_J <: B_J
  }{
    \req A_J <: \req B_J \;\;\;\; \opt A_J <: \opt B_J \;\;\;\; \seq A_J <: \seq B_J
  }
\]

\[
  \inference[\textsc{S-Extends}]{
    \class{A_F}{B}{\dots}
  }{
    \req A_F <: \req \bar{B} \;\;\;\; \opt A_F <: \opt \bar{B} \;\;\;\; \seq A_F <: \seq \bar{B}
  }
\]
\caption{Basic subtyping rules}
\label{basic-subtype}
\end{figure}


\subsection{Operational semantics}

We offer a small-step operational semantics for portions of the
evaluation for sequence indexing, class member selection, and function
invocation that illustrate the conditions under which these operations
evaluate to \<null>.  Where these rules do not apply, evaluation
proceeds under the expected interpretation (choose the $i^{th}$
element of the sequence, fetch the instance variable, invoke the
method.)  We follow the convention that the metavariables $v$ and $w$ 
represent final values, where the other lower case metavariables represent
arbitrary terms.  

\subsubsection{Sequence indexing}

Figure \ref{index-semantics} shows the small-step operational
semantics for sequence element expressions of the form $x[i]$.  Informally, they
require that $x$ be evaluated before $i$, and that index expressions
where the sequence or index is \<null> evaluate to \<null>.

\begin{figure}[htpb]
\[ \inference{e \mapsto e'}{e[i] \mapsto e'[i]} \]

\[ \inference{v \texttt{ != } \<null> & i \mapsto i'}{v[i] \mapsto v[i']} \]

\[ \inference{}{\<null>[i] \mapsto \<null>} \]

\[ \inference{}{v[\<null>] \mapsto \<null>} \]

\[ \inference{w < 0 \texttt{ || } w >= \sizeof v}
{v[w] \mapsto \<null>} \]

\caption{Operational semantics for sequence indexing}
\label{index-semantics}
\end{figure}

\subsubsection{Function invocation}

Figure \ref{function-semantics} shows the operational semantics for
function invocation.  Informally, it requires that the function
reference be evaluated before the argument, and that invoking a \<null>
function value evalutes to \<null>.

\begin{figure}[htpb]
\[ \inference{f \mapsto f'}{f(x) \mapsto f'(x)} \]

\[ \inference{x \mapsto x' & v != \<null>}{v(x) \mapsto v(x')} \]

\[ \inference{}{\<null>(x) \mapsto \<null>} \]
\caption{Operational semantics for function application}
\label{function-semantics}
\end{figure}

\subsubsection{Member selection}

Figure \ref{member-semantics} shows the operational semantics for
class member selection.  Informally, it shows that member selection on
\<null> object references evalutes to \<null>.

\begin{figure}[htpb]
\[ \inference{a \mapsto a'}{a.x \mapsto a'.x} \]

\[ \inference{}{\<null>.x \mapsto \<null>} \]
\caption{Operational semantics for member selection}
\label{member-semantics}
\end{figure}


\section{The user type system}

Not all types in the abstract type system are denotable or expressible
in the user type system.  Figure \ref{user-types} shows the mapping of
type names in the user type system to their corresponding concrete
types.  

\begin{figure}[htpb]
\begin{eqnarray*}
  \<Integer> &:=& \req \<Int32> \\
  \<Number> &:=& \req \<Float64> \\
  \<Boolean> &:=& \req \<Boolean> \\
  \<String> &:=& \req \texttt{java.lang.String} \\
  \<Duration> &:=& \req \texttt{javafx.lang.Duration} \\
  \<function>(T) : U &:=& \opt (T -> U) \\
  G &:=& \opt G \;(\textrm{except as above}) \\
  G\sqb &:=& \seq G \\
  \texttt{Required<G>} &:=& \req G \textrm{ (proposed)}\\
  \texttt{Optional<G>} &:=& \opt G \textrm{ (proposed)}\\
  \texttt{Sequence<G>} &:=& \seq G \textrm{ (proposed)}\\
\end{eqnarray*}

\caption{User type system}
\label{user-types}
\end{figure}

The name mappings are chosen so that the default for numbers, strings,
and booleans is non-nullable, while the defaults for other types is
nullable.  (Most imperative languages, such as Java and C\#, treat
numbers specially in this way.)  However, it is desirable to allow
users to denote any of the cardinalities for a given ground type; 
figure \ref{user-types} shows some proposed syntax for this.


\section{Java primitive types}

While the ``built-in'' types \<Integer> and \<Number> are generally
enough for standalone programs, since JavaFX classes can extend Java
classes, the type system needs to accomodate Java primitives.  We can
easily add new ground types for the additional Java primitives
(\<Int8>, \<Int16>, \<UInt16>, \<Int64>, and \<Float32>) to the abstract type system: 

\[
     P ::= \<Int8> \alt \<Int16> \alt \<UInt16> \alt \<Int32> \alt \<Int64> \alt \<Float32> \alt \<Float64> \alt \<Boolean> 
\]

Similarly, we can add names for these types to the user type system as we defined \<Integer>
and \<Boolean>.  We will use the convention \<JavaXxx> where Xxx is the name of the associated wrapper class (e.g., \<Byte>, \<Double>, etc) to represent non-nullable (required) types in the user type
system.  \<Number> would continue to be a synonym for \<Float64>, preserving the desirable subtyping relationship

\[ \<Integer> <: \<Number> \]

\subsection{Primitives in Java type signatures}

JavaFX code can access Java fields and call Java methods.  Where a Java
type signature (method or field) contains a primitive type, JavaFX interprets this 
as a required type (e.g., a \<float> is interpreted has having type $\req \<Float32>$.)  
Reference types in Java type signatures are treated as optional types.  
For example, when assigning the
result of a Java method that return \<String> to a JavaFX variable, the type
of the method return value is $\opt \<String>$, and so it would have to be converted
to a value of type $\req \<String>$ as described in section \ref{sec:null-conversions}.  

\subsection{Primitive conversions}

We introduce subtyping relationships corresponding to the
widening conversions for primitive types in Java:

\[ \<Int8> <: \<Int16> <: \<Int32> <: \<Int64> \]
\[ \<UInt16> <: \<Int32> \]
\[ \<Int32> <: \<Float64> \]
\[ \<Float32> <: \<Float64> \]

We also introduce (lossy) narrowing conversions for the opposite
direction, possibly with a compiler warning:

\[ \<Int64> \convertsto \<Int32> \convertsto \<Int16> \convertsto \<Int8> \]
\[ \<Int32> \convertsto \<UInt16> \]
\[ \<Float64> \convertsto \<Int32> \]
\[ \<Float64> \convertsto \<Float32> \]


\subsection{Java arrays}

While it is possible to convert between Java arrays and JavaFX
sequences, the conversion is expensive.  To reduce the need for
conversion, we introduce array types into the type system.  Where
array types appear in Java type signatures, they are treated as array
types rather than converting to sequences.  

We can introduce arrays into the JavaFX type system by slighly refactoring the 
syntax of ground types and adding an $\arr$ type constructor, introducing (possibly nested)
arrays of primitive types and reference types: 

\begin{eqnarray*}
      G_{J} &::=& P \alt C \alt \arr G_{J} \\
      G &::=& G_{J} \alt T -> S \\
\end{eqnarray*}

The corresponding form in the user type system for an array of type $G_{J}$ would be
\texttt{JavaArray<$G_{J}$>}:

\[
  \texttt{JavaArray<$G_{J}$>} := \opt \arr G_{J} \textrm{ (proposed)} \\
\]

Accesses to array variables
would be restricted to element-fetch, element-set, and \<sizeof>,
rather than the full set of sequence operations.  (Note that in this
formulation arrays are denotable but not expressible, making them
largely useful only for accessing existing Java arrays and overriding
Java methods that take array arguments.  Given that the purpose of
introducing arrays into the type system is largely for
interoperability, this may be an acceptable restriction.)  
Partial typing rules for Java arrays are shown in figure \ref{fig:arrays}.  
The operational semantics for Java arrays are not yet determined; we may
want to throw NPE and AIOOBE for erroneous index expressions instead of
evaluating to \<null>.  

\begin{figure}[htbp]

\[
\inference[\textsc{T-PrimArr}]{
  \Gamma |- x : \req \arr P & \Gamma |- t : \req \<Int32>
}{
  \Gamma |- x[t] : \req P
}
\]

\[
\inference[\textsc{T-Arr}]{
  \Gamma |- x :\req \arr C & \Gamma |- t : \req \<Int32>
}{
  \Gamma |- x[t] : \opt C
}
\]

\caption{Typing rules for Java arrays}
\label{fig:arrays}
\end{figure}


%% Denotable / expressible
%% Void

%% \section{Maps}

\end{document}
